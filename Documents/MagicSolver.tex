\textbackslash documentclass{[}12pt,a4paper{]}\{article\}
\textbackslash usepackage{[}utf8{]}\{inputenc\}
\textbackslash usepackage{[}italian{]}\{babel\}
\textbackslash usepackage\{amsmath\}
\textbackslash usepackage\{amsfonts\}
\textbackslash usepackage\{enumitem\}
\textbackslash usepackage\{geometry\}
\textbackslash geometry\{margin=2.5cm\}

\% Rimuove l\textquotesingle indentazione automatica dei paragrafi per
un look più moderno
\textbackslash setlength\{\textbackslash parindent\}\{0pt\}
\textbackslash setlength\{\textbackslash parskip\}\{1em\}

\textbackslash title\{\textbackslash textbf\{Progetto MagicSolver\}\}
\textbackslash author\{\} \textbackslash date\{\}

\textbackslash begin\{document\}

\textbackslash maketitle

\% Generazione automatica dell\textquotesingle indice
\textbackslash tableofcontents \textbackslash newpage

\textbackslash section\{Introduzione\} Il Cubo di Rubik è il rompicapo
più famoso della storia, icona intramontabile e simbolo degli anni
\textquotesingle80. Venne inventato dall\textquotesingle architetto e
professore ungherese \textbackslash textbf\{Ernő Rubik nel 1974\}. Si
tratta di un poliedro regolare presentante 6 facce, ognuna con un colore
differente, composte da 9 quadratini organizzati in una griglia \$3
\textbackslash times 3 \textbackslash times 3\$.

Negli anni sono nate tantissime variazioni del cubo classico, come il
\textbackslash textit\{Mirror Cube\}, il
\textbackslash textit\{Pyraminx\} o il più essenziale \$1
\textbackslash times 1 \textbackslash times 1\$, ma la versione più
iconica rimane il \$3 \textbackslash times 3 \textbackslash times 3\$.
Dal punto di vista computazionale, rappresenta un problema combinatorio
di elevata complessità: ogni rotazione genera un nuovo stato del
sistema. Il numero totale di configurazioni possibili è di circa \$4.3
\textbackslash times 10\^{}\{19\}\$.

Con uno spazio degli stati così vasto e un unico stato obiettivo
(\textbackslash textit\{goal state\}), può
un\textquotesingle intelligenza artificiale risolvere questo problema in
modo efficiente? Il progetto \textbackslash textbf\{MagicSolver\} si
occupa di rispondere a questo dilemma.

\textbackslash section\{Descrizione dell\textquotesingle agente\}

\textbackslash subsection\{Obiettivi\} Lo scopo di questo progetto è
creare un\textquotesingle IA capace di
``giocare\textquotesingle\textquotesingle{} con un Cubo di Rubik e di
risolvere i rompicapo lasciati in sospeso per anni sugli scaffali.

La sfida principale risiede nell\textquotesingle ampiezza dello spazio
degli stati; poiché esiste un solo stato di risoluzione, è estremamente
improbabile che il cubo venga completato attraverso mosse casuali senza
l\textquotesingle ausilio di un algoritmo strutturato. Sviluppare un
agente in grado di navigare questo spazio combinatorio fornisce
importanti intuizioni sulla risoluzione di problemi complessi ad ampia
scala.

\textbackslash subsection\{Specifica PEAS\} Per definire
l\textquotesingle architettura dell\textquotesingle agente, utilizziamo
il framework PEAS (\textbackslash textit\{Performance, Environment,
Actuators, Sensors\}):

\textbackslash begin\{itemize\} \textbackslash item
\textbackslash textbf\{Performance\}: L\textquotesingle algoritmo viene
valutato in base al numero totale di mosse eseguite per raggiungere lo
stato finito. \textbackslash item \textbackslash textbf\{Environment\}:
L\textquotesingle insieme di tutti i possibili stati del cubo.
\textbackslash item \textbackslash textbf\{Actuators\}: Le rotazioni
possibili (orarie e antiorarie) per ogni riga e colonna del cubo.
\textbackslash item \textbackslash textbf\{Sensors\}: Interfaccia con il
cubo di rubik scomposto in una rappresentazione 2D.
\textbackslash end\{itemize\}

\textbackslash subsubsection\{Caratteristiche
dell\textquotesingle ambiente\} \textbackslash begin\{itemize\}
\textbackslash item \textbackslash textbf\{Singolo agente\}: Solo
l\textquotesingle IA agisce sul cubo. \textbackslash item
\textbackslash textbf\{Totalmente osservabile\}: Ogni faccia e colore
sono visibili all\textquotesingle agente. \textbackslash item
\textbackslash textbf\{Deterministico\}: Ogni azione porta a uno stato
unico e certo. \textbackslash item \textbackslash textbf\{Statico\}:
L\textquotesingle ambiente non cambia mentre l\textquotesingle agente
sta deliberando. \textbackslash item
\textbackslash textbf\{Sequenziale\}: La mossa attuale influenza le
configurazioni future. \textbackslash end\{itemize\}

\textbackslash subsection\{Analisi del problema\} \% Qui puoi inserire i
dettagli tecnici sulla ricerca nello spazio degli stati o
l\textquotesingle algoritmo scelto (es. A*, BFS, Kociemba)
L\textquotesingle analisi si concentra sulla navigazione di un grafo in
cui i nodi rappresentano le configurazioni del cubo e gli archi
rappresentano le rotazioni. Data la dimensione dello spazio degli stati,
la ricerca cieca è impraticabile, richiedendo l\textquotesingle uso di
euristiche o algoritmi di ricerca informata.

\textbackslash end\{document\}
